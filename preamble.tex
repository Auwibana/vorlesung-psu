%*****************************************************************************
%** beamer setup  **************************************************************

\usecolortheme{rose}
\usetheme[wide=true]{sra}
\setbeamertemplate{navigation symbols}{}

\usepackage[physicalpagesinpdftoc,sracolors,autonotes]{beamertools}


\usepackage{pageslts} %  theCurrentPage
\usepackage{calc} % theCurrentPage
\usepackage{csquotes}
\usepackage{booktabs}
\usepackage{xspace}
\usepackage{multirow}
\usepackage{pgffor}

%\usepackage[utf8]{inputenc}
%\usepackage[T1]{fontenc}

\usepackage{lmodern}
\usepackage[scaled=0.85]{beramono}

% Section Title and Block Number
\newcommand{\psuSectionStart}[2]{
  \ifdefstring{\psuOldTitle}{#1}
  {}% Then
  {\section{#1}}%
  \def\psuOldTitle{#1}
}


\newwrite\cutfile
\immediate\openout\cutfile=\jobname.topics
\newcounter{FirstPage} % counter, for first page number
\newcounter{EndPage} % counter, for first page number
\setcounter{FirstPage}{1} % Set FirstPage to page 1
\newcommand{\psuSectionStop}[2]{% {topic}{lines}
  \setcounter{EndPage}{\theCurrentPage-1}
  \immediate\write\cutfile{\theFirstPage -\theEndPage:#2:#1}%
  \typeout{Topic: \theFirstPage -\theEndPage:#2:#1}
  \setcounter{FirstPage}{\theCurrentPage}% set FirstPage to new first page
}

% Parse Filename
\def\retrieveFileName#1-#2.#3\next{%
  \gdef\fileNumber{#1}%
  \gdef\fileName{#2.#3}%
  \gdef\fileTopic{#2}%
  \gdef\fileMode{#3}%
}%
\expandafter\retrieveFileName\jobname\next\relax%

\input{preamble.\fileMode.tex}

\AtBeginDocument{
  \pagenumbering{arabic} % for pageslts
  \setcounter{part}{\fileNumber\relax}
}

% Name, token and semester of the course
\def\xtoken{PSÜ}
\def\xtitle{Programmiersprachen und Übersetzer (PSÜ)}%
\def\xterm{SS}
\def\xyear{20}

  % General organization of the course
  \def\PartA{Einführung}
  \def\PartB{Prozesse und Dateien}
  \def\PartC{Interaktion und Kommunikation}
  \def\PartD{Speicher und Zugriffsschutz}
  \def\PartX{Anhang}

  \institute[SRA]{Institute for Systems Engineering\\System- und Rechnerarchitektur (SRA)}
  \author[\copyright\,cd]{Christian Dietrich}

  \def\termurl{\url{https://sra.uni-hannover.de/Lehre/\xterm\xyear/V_\xtoken}}
  \edef\term{\xterm\, \xyear}
  \def\auxWS{Wintersemester}
  \def\auxSS{Sommersemester}
  \edef\TERM{\csuse{aux\xterm} 20\xyear}

  \def\xshortdate{\ifx\insertpart\empty\term{}\else Teil~\thepart, \term{}\fi}
  \date[\xshortdate]{\TERM}

  \title[\xtoken]{\xtitle}

  % Customize numbering:
  % We want to number frames as <chapter>-<frame> with <frame> being by-chapter numbers
  % Note: The following works only if a patch is applied to beamerbaseframe.sty (see README)
  \renewcommand{\theframenumber}{\thepart--\arabic{framenumber}}
  \renewcommand{\insertframenumber}{\theframenumber}
  \renewcommand{\InsertFrameNumber}{\insertframenumber}

  \newcommand{\columntitle}[1]{
    {\centering\structure{\strut #1}\par}
  }


  \newcommand{\animation}[3][]{%
  \foreach \p/\s in {#2} {%
    \includegraphics<\s|handout:\s|skript:\s>[page=\p,#1]{#3}%
  }%
}


%*****************************************************************************
%** bibliography
%*************************************************************
\usepackage{polyglossia}
\setdefaultlanguage{german}

\usepackage[style=numeric-comp,hyperref,maxnames=3,minnames=3,defernums=true]{biblatex}
\defbibheading{bibliography}{}
\bibliography{refs}
\renewcommand{\bibfont}{\scriptsize}
\DeclareFieldFormat{postnote}{#1}



%****************************************************************
%** tikz stuff
\usepackage{tikz}
\usetikzlibrary{tikzmark,arrows,fit,calc,positioning,sra}
\usepackage{pgfkeys}
\tikzset{
  >=latex',
  small tree/.style={
    every node/.append style={
      font=\footnotesize,
      draw,
    },
    level distance=1cm,
  }
}

%*****************************************************************************
%** Legalcode *************************************************************

\pgfkeys{
  /legalcode/commons/.style={url={https://commons.wikimedia.org/wiki/File:#1}},
  /legalcode/url/.code={%
    \def\lcText##1{\href{#1}{##1}}
  },
  /legalcode/author/.code={\def\lcAuthor{, #1}},
  /legalcode/license/CC BY-SA 3.0/.code={%
    \def\lcLicense{\href{https://creativecommons.org/licenses/by-sa/3.0/legalcode}{CC BY-SA 3.0}}%
  },
  /legalcode/license/Free Art License/.code={%
    \def\lcLicense{Free Art License}%
  }
}
\newcommand{\legalcode}[3][]{% [opts]{license}{title}
  \bgroup%
  \def\lcText##1{##1}
  \def\lcAuthor{}
  \pgfkeys{/legalcode/.cd,#1,license/#2}%
  \parbox{\textwidth}{\centering\color{gray}\tiny\lcText{#3}\lcAuthor{} \mbox{(\lcLicense)}}
  \egroup
}

%*****************************************************************************
%** code blocks **************************************************************

  \makeatletter
  \setbeamercolor{code}{fg=black,bg=codecolor}

  \define@key{beamercolbox}{text width}{\pgfmathsetlengthmacro{\beamer@colbox@wd}{#1}}
  \define@key{beamercolbox}{text height}{\pgfmathsetlengthmacro{\beamer@colbox@ht}{#1}}
  \define@key{beamercolbox}{text depth}{\pgfmathsetlengthmacro{\beamer@colbox@dp}{#1}}
  \define@key{beamercolbox}{background color}{\setbeamercolor{code}{fg=black,bg=#1}}
  \define@key{beamercolbox}{scale content}{\def\bt@innerscale{#1}}
  \define@key{beamercolbox}{tag}{\def\bt@tag{#1}}

  \newenvironment<>{code}[1][\empty]{%
    \begin{onlyenv}#2%
    \vspace{2pt}%
    \setkeys{beamercolbox}{#1}
    \lstset{aboveskip=0pt,belowskip=0pt,linewidth=\beamer@colbox@wd}%
    \rule{2pt}{0pt}\begin{beamercolorbox}[colsep*=2pt,text width=\linewidth-4pt, text depth=0ex,#1]{code}%
    \ifdef\bt@innerscale{%
      \pgfmathsetlength{\@tempdima}{\beamer@colbox@wd/\bt@innerscale}
      \begin{adjustbox}{scale=\bt@innerscale}
      \begin{minipage}{\@tempdima}
        \lstset{linewidth=\@tempdima}%
      }{}
  }{%
    \ifcsdef{bt@tag}{%
      \vspace{-1.2\baselineskip}\hfill{\emph{\color{black!70!white}\csuse{bt@tag}}}
      \vspace{0.4\baselineskip}
    }{}%
    \ifdef\bt@innerscale{%
      \end{minipage}
      \end{adjustbox}
    }{}%
    \end{beamercolorbox}\rule{2pt}{0pt}\vspace{2pt}%
    \end{onlyenv}%
    \ignorespacesafterend%
  }

  \usepackage{xparse}
  \usepackage{realboxes}

  \usepackage{lecturesourcecode}

  % Makros to underbrace/overbrace elements in listings
  %
  % \lstub<overlay spec>{stuff}{brace comment}
  % \lstob<overlay spec>{stuff}{brace comment}
  %
  %
  \newcommand<>{\lstub}[2]{\ensuremath{%
    \underset{%
      \onslide#3{\text{#1}}
    }{%
      {\only#3\underbrace{
        \text{#2}
      }}
    }
  }}
  \newcommand<>{\lstob}[2]{\ensuremath{%
    \overset{%
      \onslide#3{\text{#1}}
    }{%
      {\only#3\overbrace{
        \text{#2}
      }}
    }
  }}


%%%%%%%%%%%%%%%%%%%%%%%%%%%%%%%%%%%%%%%%%%%%%%%%%%%%%%%%%%%%%%%%
% Misc
\usepackage{convention} % texmf-local
\newcommand{\advantage}[1]{\bgroup\color{safegreen!80!black}#1{\egroup}}
\newcommand{\Advantage}[1]{\advantage{\textit{#1}}}
\newcommand{\ADVANTAGE}[1]{\advantage{\textbf{#1}}}

\newcommand{\dn}[2][]{\tikz[baseline]\node[anchor=base,circle,inner sep=1pt,draw,every dn/.try,#1]{#2};}
\renewcommand{\iiad}{\ii[\color{safegreen}\textbf{+}]}  % advantage
\renewcommand{\iida}{\ii[\color{safered}\textbf{--}]} % disadvantage
\newcommand{\lecturetag}[3][]{%
  \tikz[baseline]\node[anchor=base,draw=black,fill=#2color,#1]{\color{black}Vorlesung #3};
}

\usepackage{fig/03-common}
\usepackage{fig/04-common}
\usepackage{fig/07-common}
\usepackage{fig/08-common}


\newcommand{\tikznode}[2][]{%
  \tikz[baseline,remember picture]\node[anchor=base,#1]{#2};%
}
\newenvironment<>{tikznodeenv}[1][]{%
  \begingroup%
  \tikzset{tikznode@style/.style={#1}}%
  \begin{lrbox}{\@tempboxa}%
}{\end{lrbox}%
  \tikz[remember picture]\node[inner sep=0,outer sep=0, tikznode@style]{\usebox{\@tempboxa}};
  \endgroup%
}

\newenvironment<>{visible}[1][]{\begin{scope}[visible on=#2,#1]}{\end{scope}}

\newsavebox{\@overlaybox}
\newenvironment<>{overlaybox}[1][]{%
  % We set the overlay tikz node in the next shipout routine.
  % Thereby, we do not produce a \leavemode after the lrbox
  \only#2{%
    \AtBeginShipoutNext{%
      \AtBeginShipoutUpperLeftForeground{%
        \begin{tikzpicture}[remember picture, overlay]%
          \node[fill=white,draw] at (current page.center) [#1] {\usebox{\@overlaybox}};
        \end{tikzpicture}%
      }%
    }%
  }%
  % Produce the @conclusionoverlay
  \setbox\@overlaybox=\vbox\bgroup%
}{%
  \egroup%
  \global\setbox\@overlaybox=\copy\@overlaybox\relax%
}

%\usepackage{default}
\makeatletter
\newif\ifOnBeamerModeTransition
\newcommand{\slideselection}{1-}%
\newenvironment{handoutframeselect}[1][1-]{%
  \begingroup%
  \mode<handout>{%
    \gdef\beamer@currentmode{beamer}%
    \OnBeamerModeTransitiontrue%
    \renewcommand{\slideselection}{#1}}%
}{%
  \ifOnBeamerModeTransition%
    \OnBeamerModeTransitionfalse%
    \gdef\beamer@currentmode{handout}%
  \fi%
  \endgroup%
}
\makeatother
