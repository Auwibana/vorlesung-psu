\documentclass{lecturefig}
\usetikzlibrary{trees}
\begin{document}
\begin{frame}<1-10>
  \begin{tikzpicture}[
    level distance=1cm,
    level 1/.style={sibling distance=2cm},
    level 2/.style={sibling distance=1cm},
    ]

  \begin{scope}[every node/.style={draw,fill=white}]
  \node (root) {+}
    child {node (mul) {*}
      child {node (c1) {1}}
      child {node (c2) {2}}
    }
    child {node (div) {/}
    child {node (c3) {3}}
      child {node (c4) {4}}
    };
  \end{scope}

  % Wurzel, Innerer Knoten, Blatt
  \begin{scope}[visible on=<2>, shorten <=5pt]
    \draw[ultra thin] (root) -- ++(0:2cm) node[anchor=west]{Wurzel};
    \draw[ultra thin] (div) -- ++(0:1cm) node[anchor=west]{Innerer Knoten};
    \draw[ultra thin] (c4) -- ++(0:1cm) node[anchor=west]{Blatt};
  \end{scope}


  % Baum und Unterbaum
  \begin{scope}[visible on=<3>]
    \draw[dashed,thick,srared] \convexpath{c4,c1,root}{20pt};

    \begin{scope}[on layer=background]
      \fill[luhblue!40] \convexpath{c4,c3,div}{15pt};
    \end{scope}

    \draw[shorten <=17pt] (root) -- ++(0:2cm) node[anchor=west]{Baum};
    \draw[shorten <=9pt] (div) -- ++(0:1.5cm) node[anchor=west,align=center]{(Unter-)\\Baum};
  \end{scope}


\end{tikzpicture}
\end{frame}

\begin{frame}<1-10>
  \begin{tikzpicture}[
    level distance=1cm,
    level 1/.style={sibling distance=2cm},
    level 2/.style={sibling distance=1cm},
    ]
    \begin{scope}[every node/.style={draw,fill=white},visible on=<1>]
      \node (root) {+}
         child {node (P){+}
           child {node {+}
             child {node {1}}
             child {node {2}}
           }
           child {node {3}};
         }
         child {node {4}};
         \draw (root) -- (P); % I have no idea why i need this

         \node [below=3.2cm of root,draw=none] {Linksableitung};

    \end{scope}
    \begin{scope}[every node/.style={draw,fill=white},visible on=<2>]
      \node (root) {+}
      child { node (P) {+}
        child {node {1}}
        child {node {2}}
      }
      child {node {+}
        child {node {3}}
        child {node {4}}
      };
    \end{scope}

    \begin{scope}[every node/.style={draw,fill=white},visible on=<3>]
      \node (root) {+}
         child {node {1}}
         child {node {+}
           child {node {2}}
           child {node {+}
             child {node {3}}
             child {node {4}}
           }};

      \node [below=3.2cm of root,draw=none] {Rechtsableitung};
    \end{scope}

  \end{tikzpicture}
\end{frame}
\end{document}
