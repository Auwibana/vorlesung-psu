\documentclass{lecturefig}
\usepackage{calc}
\begin{document}
\newlength\lecturedot
  \setlength\lecturedot{\widthof{\textbf{V12}}}
\begin{tikzpicture}
  \tikzset{
    topic/.style={
      draw,align=left
    },
    lecture/.style={
      label={[circle,anchor=mid,draw,fill=black,inner sep=2pt,align=center,
        text width=\lecturedot,
        font=\color{white}]north east:\textbf{V#1}},
    },
  }
  \def\Title#1{\textbf{\Large #1}}
  \begin{scope}[
    topic/.append style={
      text width=6cm,
      anchor=north west,
    },
    principle/.style={
      fill=badbee!40,
    },
    paradigm/.style={
      fill=luhblue!40,
    },
  ]
  \node[topic,lecture=3,principle] (type) {
    \Title{Typen}\\
    Primitive Typen\\
    (Un)Strukturierte Komposition\\
    Funktionstypen\\
    Subtyping\\
    Traits \& Typklassen\\
  };
  \node[right=of type.north east,topic,principle,lecture=4] (name) {
    \Title{Namen}\\
    Statische und Dynamische Bindung\\
    Scoping Rules\\
    Type-directed Name Resolution\\
    Overloading\\
    Static and Dynamic Dispatch
  };
  \node[below=of type.south west,topic,principle, lecture=7] (operation) {
    \Title{Operationen}\\
    Standardkonstrukte\\
    Auswertungsstrategie\\
    Call-By-\{Val, Ref, Name, Need\}\\
    Seiteneffekte\\
    Verhaltenskomposition
  };

  \node[right=of operation.north east,topic,principle,lecture=6] (object) {
    \Title{Objekte}\\
    Data Abstraction\\
    Erzeugung \& Initialisierung\\
    Lebenszeiten \& Ownership\\
    Mixin Vererbung\\
    Prototype Vererbung
  };

  \node[left=of operation.north west,yshift=5mm,topic,anchor=east,paradigm,lecture=13] (function) {
    \Title{Das Funktionale Paradigma}\\
    Parametrische Datentypen\\
    Unifikation \& Typinferenz\\
    Pattern Matching\\
    Monaden\\
    Immutable Data
  };

  \node[right=of object.north east,topic,yshift=5mm,anchor=west,paradigm,lecture=12] (oo) {
    \Title{Das Objektoriente Paradigma}\\
    Klassenbasierte Vererbung\\
    Kapselung von Daten und Verhalten\\
    Module und Namensräume\\
    SOLID-Prinzipien\\
  };

  \node[left=of function.west,align=left,font=\Large\scshape]{
    S\\[.5ex]
    P\\[.5ex]
    R\\[.5ex]
    A\\[.5ex]
    C\\[.5ex]
    H\\[.5ex]
    E\\[.5ex]
    N
  };
\end{scope}

\begin{scope}[
    topic/.append style={
      text width=5.5cm,
      anchor=north west,
      fill=safegreen!40,
    }
  ]


\node[below=2 of operation.south west,topic,lecture=5] (semantic) {
    \Title{Semantische Analyse}\\
    Typprüfung\\
    Namensauflösung\\
    Fehlermeldungen\\
    C-Präprozessor/M4\\
    Hygenische Makros
  };

  \node[left=1.5 of semantic.north west,topic,anchor=north east,
       lecture=2,
       ] (parsing) {
    \Title{Syntaktische Analyse}\\
    Tokenstrom\\
    (Kontextfreie) Grammatiken\\
    Abstrakter Syntaxbaum\\
    Parsertechniken\\
    Parsergeneratoren
  };

    \node[left=of parsing.north west,anchor=north east,align=left,font=\Large\scshape](label-compiler){
    Ü\\[.5ex]
    B\\[.5ex]
    E\\[.5ex]
    R\\[.5ex]
    S\\[.5ex]
    E\\[.5ex]
    T\\[.5ex]
    Z\\[.5ex]
    E\\[.5ex]
    R
  };

  \node[right=1.5 of semantic.north east,topic,lecture=8] (intermediate) {
    \Title{Zwischencode}\\
    3-Address Code\\
    Code für Arith. Ausdrücke\\
    R-Values \& L-Values\\
    Basic Blocks \& Kontrollflussgraph\\
    Kontrollflusskonstrukte
  };

  \node[right=1.5 of intermediate.north east,topic,lecture=10] (machine) {
    \Title{Maschinencode}\\
    Registerallokation\\
    Funktionsaufrufe\\
    Application Binary Interface\\
    Peephole Optimizer
  };

  \node[below=1.5 of intermediate.south west,topic,lecture=9] (opt) {
    \Title{Optimierung}\\
    Fixpunkt Analyse\\
    Konstantenfortschreibung\\
    Konstante Auswertung\\
    Entfernung toten Codes
  };

  \node[right=1.5 of opt.north east,topic,lecture=11] (runtime) {
    \Title{Das Laufzeitsystem}\\
    ELF und Linker\\
    Dynamische Bibliotheken\\
    Standardbibliotheken\\
    Garbage Colletion\\
    Just-in-Time Compilation
  };

  \draw[->] (parsing) -- node[above]{AST} (semantic);

  \draw[->] (semantic) -- node[above]{AST} (intermediate);

  \draw[->] (intermediate) to[bend left] (opt);
  \node at ($(intermediate.south)!0.5!(opt.north)$) {IR};
  \draw[->] (opt) to[bend left] (intermediate);
  \draw[->] (intermediate) -- node[above]{IR} (machine);


  \node at ($(machine.south east)!0.5!(runtime.north east)$) (bin) {\textsc{Binärprogramm}};
  \draw[->] (machine.south) to[out=-90,in=180] (bin);
  \draw[->] (runtime.north) to[out=90,in=180] (bin);

\end{scope}

\coordinate (label-anchor) at ($(operation.south)!0.5!(semantic.north)$);
\draw[dashed] (label-anchor-|label-compiler) -- (label-anchor-|bin.east);

\end{tikzpicture}
\end{document}
