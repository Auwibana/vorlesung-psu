\documentclass{lecturefig}
\usepackage{fig/04-common}
\begin{document}
\begin{frame}[fragile]
  \begin{tikzpicture}[
    level 1/.style={sibling distance=2.5cm},
    level 2/.style={sibling distance=2cm},
    ]
    \begin{scope}[
      every node/.style={
        draw,execute at begin node={\strut}
      },
      edge from parent/.style={draw,<-},
      ]
    \node[onslide=<2>{fill=letcolor}] (B) {B}
      child { node (S1) {S1}
        child { node[onslide=<2>{fill=refcolor}] (S11) {S11} }
        child { node (S12) {S12} }
      }
      child { node (S2) {S2}
        child { node (S22) {S22} }
      }
      ;
      \only<2>{
        \draw[ultra thick,->,srared](S11) -- (S1);
        \draw[ultra thick,->,srared](S1) -- (B);
      }
    \end{scope}
    \node[above=0.2 of B] (label-vererbung) {\structure{Vererbungshierarchie}};


    % Overload Set
    \begin{scope}[
      start chain=going right,
      node distance=1mm,
      code/.style={font=\ttfamily},
      arg/.style={draw, fill=white,on chain,code,minimum width=1cm}
      ]

      \node[right=2 of label-vererbung] {\structure{Overload Set}};

    % F1
    \node[code,on chain] at ($(B)+(east:3cm)$) (f1) {f(};
    \def\arglist#1#2#3{
      \node[arg,arg1/.try] (arg1) {#1};
      \node[on chain,inner sep=0,]{\strut,};
      \node[arg,arg2/.try] (arg2) {#2};
      \node[on chain,inner sep=0,]{\strut,};
      \node[arg,arg3/.try] (arg3) {#3};
      \node[code,on chain] (paren) {)};
    }
    {\tikzset{arg1/.style={onslide=<2>{fill=letcolor}}}
      \arglist{\alt<-7>{B}{S1}}{\alt<-7>{B}{S2}}{B}
    }

    \begin{scope}[visible on=<2->]
      \foreach \arg/\score in {1/\alt<-7>{2}{1}, 2/\alt<-7>{1}{0}, 3/0} {
        \draw[->](arg\arg) -- ++(down:1.5em) node[below] (score\arg) {\score};
      }
      \draw[->,double] (score3 -| paren.east) -- ++(east:1em) node[right,text width=1.3cm](score-f1){$\sum=\alt<-7>{3}{1}$};
    \end{scope}

    % F2
    \node[code,on chain,below=1.6 of f1.west,anchor=west]  (f2) {f(};
    {\tikzset{arg3/.style={onslide=<3-5>{fill=srared!50}}}
      \arglist{S11}{B}{int}
    }
    \begin{scope}[visible on=<4->]
      \foreach \arg/\score in {1/0, 2/1, 3/$\infty$} {
        \draw[->](arg\arg) -- ++(down:1.5em) node[below] (score\arg) {\score};
      }
      \draw[->,double] (score3 -| paren.east) -- ++(east:1em) node[right,text width=1.3cm](score-f2){$\sum=\infty$};
    \end{scope}

    % F3
    \node[code,on chain,below=1.6 of f2.west,anchor=west] (f3) {f(};
    \arglist{S11}{B}{B}
    \begin{scope}[visible on=<6->]
      \foreach \arg/\score in {1/0, 2/1, 3/0} {
        \draw[->](arg\arg) -- ++(down:1.5em) node[below] (score\arg) {\score};
      }
      \draw[->,double] (score3 -| paren.east) -- ++(east:1em) node[right,text width=1.3cm](score-f3){$\sum=1$};
    \end{scope}

    \begin{pgfonlayer}{background}
      \node[fit=(f1) (score-f1),inner sep=0,visible on=<3-8>,fill=badbee!30] (f1-box) {};
      \node[fit=(f2) (score-f2),inner sep=0,visible on=<5-7>,fill=srared!30] (f2-box){};
      \node[fit=(f3) (score-f3),inner sep=0,visible on=<{6,8}>,  fill=badbee!30] (f3-box){};
      \node[fit=(f3) (score-f3),inner sep=0,visible on=<7>, fill=safegreen!30] (choosen) {};
    \end{pgfonlayer}
  \end{scope}

  \node[anchor=north,yshift=-1em] at (B.center|-S11.south) (label-callsite) {\structure{Aufrufstelle}};
  \node[below=0 of label-callsite,draw,font=\ttfamily] (callsite){f(\alt<2>{\colorbox{refcolor}}{\colorbox{white}}{S11}, S2, B)};
  \only<7>{\draw[->,ultra thick,safegreen](choosen.south) |- (callsite);}

  \begin{scope}[visible on=<8>,draw,srared,ultra thick,]
    \node[draw,fill=srared!20,align=left,text width=5cm,inner sep=1em,left=4mm of f2-box,font=\color{black}] (error){
      Wenn zwei Funktionen den gleichen Score haben, so wirft der Übersetzer einen Fehler.
    };
    \draw[->] (f1-box.west) -| (error.north);
    \draw[->] (f3-box.west) -| (error.south);
  \end{scope}

  \node[anchor=east] at (label-vererbung-|choosen.east) (label-specifity) {\structure{Score}};

  \end{tikzpicture}
\end{frame}
\end{document}
