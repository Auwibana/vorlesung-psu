\documentclass{lecturefig}
\usetikzlibrary{through}
\begin{document}
\begin{frame}[fragile]
  \begin{tikzpicture}[]
    \node[object,align=left,font=\ttfamily,inner ysep=2ex] (object){
      .a = 23,\\[1ex]
      .b = "foo",\\[1ex]
      .c = \&object2
    };

    \begin{scope}[every node/.style={
        minimum width=3cm,operation,font=\ttfamily, right=1cm of object,visible on=<1>
        }
        ]
      \node[yshift=8mm] {incrementCounterOfQueue(queue\_t)};
      \node[yshift=0cm] {getNameOfQueue(queue\_t)};
      \node[yshift=-8mm] {popObjectFromQueue(queue\_t)};
    \end{scope}
    
    \begin{scope}[on layer=background,every node/.style={
        message,font=\ttfamily,minimum width=3cm, right=-5pt of object,visible on=<2>,
      }
      ]
      \node[yshift=8mm] {increment()};
      \node[yshift=0cm] (name) {getName()};
      \node[yshift=-8mm] (next) {nextObject()};
    \end{scope}
  \end{tikzpicture}
\end{frame}

\begin{frame}[fragile]
  \begin{tikzpicture}[]
    \node[object,align=left,font=\ttfamily,inner ysep=2ex] (object){
      .a = 23,\\[1ex]
      .b = "foo",\\[1ex]
      .c = \&object2
    };

    \begin{scope}[every node/.style={
        minimum width=3cm,operation,font=\ttfamily, right=1cm of object,visible on=<1>
        }
        ]
      \node[yshift=8mm] {incrementCounterOfQueue(queue\_t)};
      \node[yshift=0cm] {getNameOfQueue(queue\_t)};
      \node[yshift=-8mm] {popObjectFromQueue(queue\_t)};
    \end{scope}
    
    \begin{scope}[on layer=background,every node/.style={
        message,font=\ttfamily,minimum width=3cm, right=-5pt of object,visible on=<2->
      }
      ]
      \node[yshift=8mm] (increment) {increment()};
      \node[yshift=0cm] (name) {getName()};
      \node[yshift=-8mm] (next) {nextObject()};

    \end{scope}

    \begin{visible}<2>
      \draw[ultra thin] (next.south) -- ++(down:1cm) node[below] {Methoden sind Nachrichten};
      \draw[ultra thin] (object.north west) -- ++(120:1cm) node[anchor=south east] {Felder sind der Zustand};
    \end{visible}

    \node[yshift=10mm,visible on=<2->] at (increment.west) {\Large\structure{Das Objekt}};


    \begin{visible}<3->
    \node [] at (name.west) [circle through={($(next.south east)+(east:1ex)$)}] (circle) {};
    \foreach \W/\opt/\name in
        {-30/safegreen!30/L1,
          90/luhblue!40/L2,
          210/srared!40/L3}{
      \coordinate (@) at ($(circle.\W)+(\W:3.4cm)$);
      \draw[fill=\opt] (@) --        (circle.\W+60)
            to[bend left] (circle.\W)
            to[bend left] (circle.\W-60) -- cycle;
      \coordinate (\name) at ($(circle.\W)+(\W:1.3cm)$);
    }

    \node[align=center] at (L2) (L2) {\textbf{Sichtbarkeit}};
    \node[align=center,xshift=-4pt,yshift=-3pt] at (L1) (L1) {\textbf{Dynamischer}\\\textbf{Dispatch}};
    \node[align=center] at (L3) (L3) {\textbf{Vererbung}};
  \end{visible}


  \begin{visible}<3>
    \draw[ultra thin] (L1.east) -- ++(60:1cm) node[align=center,anchor=west,xshift=3mm] {\structure{Spätes Binden}\\Auswahl der Prozedur\\ beim Nachrichtenempfang};
    \draw[ultra thin] (L2.west) -- ++(180:1cm) node[anchor=east,align=center] {Zugriffsschutz für\\ Methoden und Felder};
    \draw[ultra thin] (L3.north west) -- ++(120:1cm) node[align=center,anchor=south] {Abstraktion\\von\\Ähnlichem};

  \end{visible}


\end{tikzpicture}
\end{frame}
\end{document}
