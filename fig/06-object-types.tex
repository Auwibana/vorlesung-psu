\documentclass{lecturefig}
\begin{document}
\begin{frame}[fragile]
  \begin{tikzpicture}[remember picture,
    object/.style={
      draw,
    },
    ]

    \node[object,pin={75:Zahlen}] (obj-int) {1};
    \newcommand\OBJ[2][]{%
      \tikz[remember picture,baseline]\node[object,anchor=base,#1] {#2};%
    }

    \node[object,right=0.5 of obj-int]   (obj-arr-1) {12};
    \node[object,right=0 of obj-arr-1] (obj-arr-2) {42};
    \node[object,right=0 of obj-arr-2] (obj-arr-3) {23};
    \node[object,fit=(obj-arr-1) (obj-arr-3),pin=-135:Arrays] (obj-arr) {};

    \node[object, align=left,right=0.8 of obj-arr.south east,anchor=south west,pin=170:Record] (obj-record){
      .x = \OBJ{1}\\
      .y = \OBJ{3}
    };

    \node[object, align=left,right=0.5 of obj-record.south east,anchor=south west,pin=30:Einbettung] (obj-embed){
      .x = \OBJ{.y = \OBJ{1}}%
    };

    \node[object, align=left,below=0.5 of obj-embed.south west,anchor=north west,pin=180:Referenz] (obj-ref) {
      \strut.x = \OBJ[name=ref,fill opacity=0]{1}
    };

    \node[object, align=left,right=0.5 of obj-ref] (obj-next) {
      \strut.y = \OBJ[]{1}
    };
    \draw[->](ref.center) node[circle,fill=black,inner sep=2pt] {} -- (obj-next);
  \end{tikzpicture}
\end{frame}
\end{document}
