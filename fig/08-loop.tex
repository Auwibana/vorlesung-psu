\documentclass{lecturefig}
\begin{document}
\begin{frame}[fragile]
  \begin{tikzpicture}[current/.style={
      label={[fit=(\tikzlastnode),
        visible on=#1,
        name=current block,
        draw,ultra thick,
        luhgreen]center:}
    },
    ]

    \node[basic block=BB0,current=<1-3>] (beforeLoop) {
      \ldots\\
      \only<3->{\ircmd{Goto} .BB1}\strut\\
    };

    

    \begin{visible}<1->
      \node[anchor=north west,xshift=2em,align=left] at (beforeLoop.north east){
        
        \begin{minipage}{7cm}
        \structure{Codeerzeugung (while-Schleife)}:
        \begin{itemize}
        \item<2-> Blöcke erstellen {
            \begin{itemize}
            \item BB1: Loop-Header Block
            \item BB2: Loop Body Block
            \item BB3: Sequenzierungsblock
            \end{itemize}
          }
        \item<3-> Eintritt in die Schleife
        \item<4-> Bedingung generiern\\\texttt{self.rvalue(whileStmt.cond)}
        \item<5-> Bedingte Kontrollflussverzeigung
        \item<6-> Loop-Body generieren\\\texttt{self.visit(whileStmt.body)}
        \item<7-> Rücksprungkante
        \item<8-> \texttt{current\_block}-Invariante wiederherstellen
        \end{itemize}
      \end{minipage}

      };
    \end{visible}

    \begin{visible}<2->
      \node[basic block=BB1,current=<4-5>,below=0.4 of beforeLoop] (loopHeader) {
        \only<4->{cond := \ircmd{Assign} \ldots}\strut\\
        \only<5->{\ircmd{IfGoto} cond, .BB2, .BB3}\strut\\
      };

      \node[basic block=BB2,current=<6>,below=0.4 of loopHeader] (loopBody) {
        \only<6->{Stmt1}\strut\\
        \only<6->{Stmt2}\strut\\
        \only<7->{\ircmd{Goto} .BB1}\strut\\
      };

      \node[basic block=BB3,current=<8>,below=0.4 of loopBody] (afterBlock) {
        \only<8>{\ldots}\strut\\
      };
  \end{visible}

  \begin{visible}<3->
    \draw[->] (beforeLoop) -- (loopHeader);
  \end{visible}
  
    \begin{visible}<4->
      \draw[->] ($(loopHeader.south east)+(up:1ex)$) -- ++(east:1em) |- ($(afterBlock.north east)+(down:1ex)$);
      \draw[->] (loopHeader) -- (loopBody);
    \end{visible}
    

    
    \begin{visible}<7->
       \draw[->] ($(loopBody.south west)+(up:1ex)$) -- ++(west:1em) |- ($(loopHeader.north west)+(down:1ex)$);
    \end{visible}


    
  \end{tikzpicture}
\end{frame}
\end{document}
