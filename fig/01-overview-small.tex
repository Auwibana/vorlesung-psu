\documentclass{lecturefig}
\usepackage{calc}
\begin{document}

\newlength\lecturedot
\setlength\lecturedot{\widthof{\textbf{V12}}}
\def\lecturedotpos{north east}
  \tikzset{
    topic/.style={
      draw,align=left
    },
    lecture/.style={
      label={[circle,anchor=mid,draw,fill=black,inner sep=2pt,align=center,
        text width=\lecturedot,
        font=\color{white}]\lecturedotpos:\textbf{V#1}},
    },
    principle/.style={
      fill=principlecolor,
    },
    paradigm/.style={
      fill=paradigmcolor,
    },
    compiler/.style={
      fill=compilercolor,
    },
    highlight on/.style={
      onslide=<#1>{label={[fit=(\tikzlastnode),draw,srared, inner sep=4pt,ultra thick,]center:}}
    },
  }
  \def\Title#1{\textbf{\Large \strut #1}}


\begin{frame}<1>[label=language]
\begin{tikzpicture}
  \begin{scope}[
    topic/.append style={
      text width=6cm,
      anchor=north west,
    },
    topic double/.append style={
      text width=13.25cm,
    },
  ]
  \node[topic,lecture=3,principle,highlight on=2, highlight on=6] (type) {
    \Title{Typen}\\
    Statische und. Dynamische Typen\\
    Gleichheit und Kompatibilität\\
    Primitive und Skalare Typen\\
    Komposition von Typen\\
    Polymorphe Typen\\
  };
  \node[right=of type.north east,topic,principle,lecture=4,highlight on=3] (name) {
    \Title{Namen}\\
    Statische und Dynamische Bindung\\
    Scoping Rules\\
    Type-directed Name Resolution\\
    Overloading\\
    Static and Dynamic Dispatch
  };
  \node[below=of type.south west,topic,principle, lecture=7,highlight on=4] (operation) {
    \Title{Operationen}\\
    Standardkonstrukte\\
    Auswertungsstrategie\\
    Call-By-\{Val, Ref, Name, Need\}\\
    Seiteneffekte\\
    Verhaltenskomposition
  };

  \node[right=of operation.north east,topic,principle,lecture=6,highlight on=5] (object) {
    \Title{Objekte}\\
    Data Abstraction\\
    Erzeugung \& Initialisierung\\
    Lebenszeiten \& Ownership\\
    Mixin Vererbung\\
    Prototype Vererbung
  };

  \node[above=of type.north west,yshift=4mm,topic,paradigm,topic double,anchor=west, lecture=13,highlight on=7] (function) {
    \Title{Das Funktionale Paradigma}
  };

  \node[below=of operation.south west,yshift=-4mm,topic,paradigm,topic double,anchor=west,lecture=12] (oo) {
    \Title{Das Objektorientierte Paradigma}
  };
\end{scope}
\end{tikzpicture}
\end{frame}



\begin{frame}<1>[label=compiler]
\begin{tikzpicture}
\begin{scope}[
    topic/.append style={
      text width=5.5cm,
      anchor=north west,
      compiler,
    }
    ]


\node[below=2 of operation.south west,topic,lecture=5,highlight on=3] (semantic) {
    \Title{Semantische Analyse}\\
    Typprüfung\\
    Namensauflösung\\
    Fehlermeldungen\\
    Sprachmakros
  };

  \node[above=1.5 of semantic.north west,topic,anchor=south west,
       lecture=2,highlight on=2
       ] (parsing) {
    \Title{Syntaktische Analyse}\\
    Scanner \& Tokenstrom\\
    (Kontextfreie) Grammatiken\\
    LL(1) \& Rekursiver Abstieg \\
    LR(1) \& Parsergeneratoren \\
    Abstrakter Syntaxbaum
  };

\node[right=1.5 of parsing.north east,topic,lecture=8,highlight on=4] (intermediate) {
    \Title{Zwischencode}\\
    3-Address Code\\
    Code für Arith. Ausdrücke\\
    R-Values \& L-Values\\
    Basic Blocks \& Kontrollflussgraph\\
    Kontrollflusskonstrukte
  };

  \node[right=1.5 of semantic.north east,topic,lecture=10,highlight on=6] (machine) {
    \Title{Maschinencode}\\
    Registerallokation\\
    Funktionsaufrufe\\
    Application Binary Interface\\
    Peephole Optimizer
  };



  \node[right=1.5 of intermediate.north east,topic,lecture=9,highlight on=5] (opt) {
    \Title{Optimierung}\\
    Fixpunkt Analyse\\
    Konstantenfortschreibung\\
    Konstante Auswertung\\
    Entfernung toten Codes
  };

  \node[right=1.5 of machine.north east,topic,lecture=11,highlight on=7] (runtime) {
    \Title{Das Laufzeitsystem}\\
    ELF und Linker\\
    Dynamische Bibliotheken\\
    Standardbibliotheken
  };


  \draw[->] (parsing) -- node[right]{AST} (semantic);

  \draw[->] (semantic) -- node[above,sloped]{AST} (intermediate);

  \draw[<->] (opt)--  node[above]{IR}  (intermediate);
  \draw[->] (intermediate) -- node[left]{IR} (machine);


  \node[yshift=-1cm] at ($(machine.south west)!0.5!(runtime.south east)$) (bin) {\textsc{Binärprogramm}};
  \draw[->] (machine.south) to[out=-90,in=180] (bin);
  \draw[->] (runtime.south) to[out=-90,in=0] (bin);
\end{scope}

\end{tikzpicture}
\end{frame}


\begin{frame}
\begin{tikzpicture}
  \tikzset{
    topic/.append style={
      text width=8cm,
      anchor=north,yshift=-5mm
    }
  }

  \def\lecturedotpos{east}

  \node[topic,compiler, lecture=2]  (syntax) {
    \Title{Syntaktische Analyse}
  };

  \node[topic,fill=lightgray,lecture=1,anchor=south,yshift=10mm]  (einleitung) at (syntax.north){
    \Title{Einleitung}
  };

  \node[topic,lecture=3,principle]  (type) at (syntax.south) {
    \Title{Typen}
  };
  \node[topic,principle,lecture=4]  (name) at (type.south) {
    \Title{Namen}
  };

  \node[topic,compiler,lecture=5]  (semantic) at (name.south) {
    \Title{Semantische Analyse}
  };

  \node[topic,principle,lecture=6]  (object) at (semantic.south) {
    \Title{Objekte}
  };

  \node[topic,principle, lecture=7]  (op) at (object.south) {
    \Title{Operationen}
  };

  \node[topic,compiler,lecture=8,anchor=north west,yshift=5mm,xshift=2cm]  (@) at (syntax.north east) {
    \Title{Zwischencode}
  };

  \node[topic,compiler,lecture=9]  (@) at (@.south) {
    \Title{Optimierung}
  };

  \node[topic,compiler,lecture=10]  (@) at (@.south) {
    \Title{Maschinencode}
  };




  \node[topic,compiler,lecture=11]  (@) at (@.south) {
    \Title{Das Laufzeitsystem}
  };

  \node[topic,paradigm,lecture=12]  (oo) at (@.south) {
    \Title{Das Objektorientierte Paradigma}
  };


  \node[topic, paradigm,lecture=13]  (@) at (oo.south) {
    \Title{Das Funktionale Paradigma}
  };



  \draw (op.south west) ++(down:1.5cm)
  node[anchor=west,compiler,draw]      (l0) {\Title{Technik: Übersetzer}};
  \node[right=1 of l0 ,principle,draw] (l1) {\Title{Basis: Sprachkonzepte}};
  \node[right=1 of l1 ,paradigm,draw] (l2) {\Title{Philosophie: Designprinzipien}};



\end{tikzpicture}

\end{frame}

\againframe<2->{compiler}

\againframe<2->{language}


\end{document}
