\documentclass{lecturefig}

\usepackage{fig/04-common}
\begin{document}
\begin{frame}
  \begin{tikzpicture}[start chain=going below,node distance=0cm]
    \begin{SCOPE}[text={\LET N=},private]{N}
      \begin{SCOPE}[text={\LETe A=},private]{A}
        \LINE[] (f1) {\LETe f1};
      \end{SCOPE}
    \end{SCOPE}
    \LINE [] (@) {};
    \begin{SCOPE}[text={\LET M=},private]{M}
      \LINE[] (import) {\IMPORT ::N::A};
      \begin{SCOPE}[text={\LET B=},private]{B}
        \begin{SCOPE}[text={\LET f2=},private]{f2}
          \LINE[] (use) {A::f1};
        \end{SCOPE}
      \end{SCOPE}
    \end{SCOPE}
    \draw (N-start.east) -- ++(20:1cm) node[align=left,anchor=west] (@) {Pakete und Klassen sind\\ per default \texttt{private}};
    \draw (A-start.east) -- (@);
    \draw (f1.east) -- ++(-20:1cm) node[align=left,anchor=west] (@) {\texttt{public} macht Namen\\ dennoch sichtbar};
    \draw (import.east) -- ++(20:1cm) node[align=left,anchor=west] (@) {Java verwendet "\texttt{.}"};

    \draw (f2-start.east) -- ++(-50:1cm) node[align=left,anchor=west] (@) {Lokale Variablen sind nicht\\von außen sichtbar};
\end{tikzpicture}
\end{frame}
\end{document}
