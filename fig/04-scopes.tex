\documentclass{lecturefig}
\usetikzlibrary{chains}
\begin{document}
\tikzset{
  line/.style={
    font=\ttfamily,
  },
  comment/.style={
    label={[anchor=east]east:\C{#1}}
  },
  text/.store in=\tikztext,
}
\def\tikztext{}

\newcommand\LINE[1][]{
  \node[text width=5cm, on chain, inner sep=2\pgflinewidth] (@) {\strut};
  \node[line, ,xshift=\theindentdepth em, text width=5cm-\theindentdepth em,
        anchor=west, at=(@.west),#1]
}
\def\C#1{\hfill\bgroup\color{gray}/* #1 */\egroup}
\newcounter{indentdepth}
\newenvironment{SCOPE}[2][]{
  \LINE[#1] (#2-start) {\tikztext\{};%
  \def\SCOPEOPTS{#1}%
  \def\SCOPENAME{#2}%
  \addtocounter{indentdepth}{1}%
}{
  \addtocounter{indentdepth}{-1}%
  \LINE (\SCOPENAME-end) {\}};%
  \node[fit=(\SCOPENAME-start) (\SCOPENAME-end),inner sep=0] (\SCOPENAME) {};
}

\begin{frame}
  \begin{tikzpicture}[start chain=going below,node distance=0cm]
    \begin{SCOPE}{A}
      \LINE[] (def) {let A};
      \LINE[]       {let B};
      \LINE[] (use) {A};
      \draw[->] (use.west) -- (use.west -| A-start.west) |- (def);
    \end{SCOPE}
  \end{tikzpicture}
\end{frame}

\begin{frame}
  \begin{tikzpicture}[start chain=going below,node distance=0cm]
    \begin{SCOPE}{A}
      \LINE[onslide=<1>{draw},comment={$A_1$}] (def-A) {let A};
      \begin{SCOPE}{B}
        \LINE[onslide=<2>{draw,comment={$A_2$}}] (def-B) {let \only<1>{B}\only<2>{A}};
        \begin{SCOPE}{C}
          \LINE[draw] (use) {A};
          \only<1>{\draw[->] (use.west) -- (use.west -| A-start.west) |- (def-A);}
          \only<2>{\draw[->] (use.west) -- (use.west -| A-start.west) |- (def-B);}
        \end{SCOPE}
      \end{SCOPE}
    \end{SCOPE}
  \end{tikzpicture}
\end{frame}
\end{document}
